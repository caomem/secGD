\chapter{Métricas}
\label{ap:metric}

Como esse texto utiliza fortemente o conceito de distância, é necessário e bem vindo que se gaste algum espaço para uma construção formal dessa ideia. A noção de distância está relacionada com o conceito de \textit{métrica}, como segue.
\\

Seja $\mathcal{X}$ um espaço vetorial $K$-dimensional sobre $\mathbb{R}$. \textit{Métrica} é uma função de dois argumentos que mapeia pares ordenados de elementos em $\mathcal{X}$ para um número real não negativo. Precisamente, para todo $x, y$ e $z$ $\in \mathcal{X}$, uma função $d(\cdot,\cdot): \mathcal{X} \times \mathcal{X} \longrightarrow \mathbf{R}$ é uma métrica se satisfaz os seguintes axiomas:

\begin{enumerate}
	\item $d(x,y) = 0$ se, e somente se, $x = y$; 
	\item $d(x,y) = d(y,x)$;
	\item $d(x,z) \leq d(x,y) + d(y,z)$;
	\item $d(x,y) \geq 0$
\end{enumerate}

Nesse trabalho, quando não é especificado qual métrica se está usando, fica implícita a utilização da \textit{Métrica Euclidiana}, definida em função da \textit{Norma Euclidiana}:

\begin{equation}\tag{Norma Euclidiana}
\forall x, y \in \mathcal{X}, d(x,y) = \lVert x-y \rVert_2 = \sqrt{\langle x, y\rangle} = \sqrt{\sum_{i = 1}^{K} (x_i-y_i)^2}.
\label{eq:normaEuclidiana}
\end{equation}
\\

O par ($\mathcal{X}, d$) é chamado \textit{espaço métrico}. A noção de métrica não depende de espaços vetoriais, donde pode ser facilmente generalizada fazendo $\mathcal{X}$ um conjunto qualquer.

\chapter{Rigidez}
\label{ap:rigid}

O conceito de rigidez em grafos é pouco simples e suficientemente extenso para que não nos caiba abordá-lo completamente nesse texto. Em contrapartida, visto que o objeto central desse texto (soluções de instâncias WSNL) tem estreita relação com a definição de \textit{rigidez global}, no que segue, apresentar-se-á satisfatoriamente esta definição. 

Para um maior aprofundamento, recomenda-se \cite{libertiEDG, rigidezGrafosEAplicacoesAnaCarlile}.
\\

O conceito de \textit{rigidez} está associado com a ideia de \textit{movimento rígido} entre as possíveis realizações de um grafo.
\begin{center}
	\begin{minipage}{0.93 \linewidth}
		\textbf{Definição:} Um \textit{framework} $\mathcal{F}$ em $\mathbb{R}^K$ é um par $(G,x)$, onde $x$ é uma realização de $G$ em $\mathbb{R}^K$.
	\end{minipage}
\end{center}

\begin{center}
	\begin{minipage}{0.93 \linewidth}
		\textbf{Definição:} Um \textit{movimento do framework} $(G,x)$ é uma função contínua $y : [0,1]\longrightarrow \mathbb{R}^{nK}$, tal que:
		\begin{itemize}
			\item $y(0) = x$;
			\item $y(t)$ é uma realização válida de $G$ para todo $t \in [0,1]$.
		\end{itemize}
	\end{minipage}
\end{center}

\begin{center}
	\begin{minipage}{0.93 \linewidth}
		\textbf{Definição:} Duas realizações $x,y$ de um grafo $G=(V,E)$ são ditos \textit{congruentes} se, para todo $u,v\in V$, tem-se $\| x_u - x_v\| = \|y_u - y_v\|$. Se $x,y$ não são congruentes, então eles são \textit{incongruentes}.
	\end{minipage}
\end{center}

\begin{center}
	\begin{minipage}{0.93 \linewidth}
		\textbf{Definição:} Uma \textit{flexão} de um framework $(G,x)$ é um deslocamento $y$ de $x$ tal que $y(t)$ é incongruente a $x$ para qualquer $t\in(0,1]$.
	\end{minipage}
\end{center}

\begin{center}
	\begin{minipage}{0.93 \linewidth}
		\textbf{Definição:} Um framework é \textit{flexível} se possuir uma flexão; do contrário, é dito \textit{rígido}.
	\end{minipage}
\end{center}

\begin{center}
	\begin{minipage}{0.93 \linewidth}
		\textbf{Definição:} Dois frameworks $\mathcal{F} = (V,E,x)$ e $\mathcal{F'} = (V',E',x')$ são \textit{equivalentes} em $\mathbb{R}^K$ se há uma bijeção entre os conjuntos de vértices $V$ e $V'$ que preservem os comprimentos das arestas associados a esses.
	\end{minipage}
\end{center}

\begin{center}
	\begin{minipage}{0.93 \linewidth}
		\textbf{Definição:} Um framework $\mathcal{F} = (G,x)$ é \textit{globalmente rígido em $\mathbb{R}^K$} se, quando $\mathcal{F}$ é equivalente a $\mathcal{F}' = (G,x')$, então $x$ é congruente a $x'$.
	\end{minipage}
\end{center}


