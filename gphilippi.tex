 \documentclass[a4paper,12pt]{article}
\usepackage[a4paper,top=3cm,bottom=2cm,left=3cm,right=3cm,marginparwidth=1.75cm]{geometry}
\usepackage[brazil]{babel}
\usepackage[T1]{fontenc}
\usepackage[utf8]{inputenc}
\usepackage{amsmath}
\usepackage{MnSymbol}
\usepackage{wasysym}
\usepackage{hyperref}
\usepackage{color}
\definecolor{Blue}{rgb}{0,0,0.9}
\definecolor{Red}{rgb}{0.9,0,0}
\usepackage{esvect}
\usepackage{graphicx}
\usepackage{float}
\usepackage{indentfirst}
\usepackage{caption}
\usepackage{blkarray}
\newcommand\Mark[1]{\textsuperscript#1}
\usepackage{pgfplots}
\usepackage{amsfonts}
\usepackage[english, ruled, linesnumbered]{algorithm2e}
\usepackage{algorithmic}
%\usepackage[toc,page]{appendix}
\newtheorem{definicao}{Definição}[section]
\newtheorem{teorema}{Teorema}[section]

\title{Geometria de Distâncias}
\author{Guilherme Philippi}
\begin{document}
\maketitle
\tableofcontents

\section{Geometria de Distâncias Euclidianas}
Apresenta-se nesta seção uma introdução a \textit{Geometria de Distâncias Euclidianas}. O nome ``Geometria de Distâncias'' diz respeito ao conceito desta geometria basear-se em distâncias ao invés de pontos. A palavra ``Euclidiana'' é importante para caracterizar as arestas --- elementos fundamentais associados as distâncias --- como segmentos, sem restringir seus ângulos de incidência \cite{libertiEDG}.

\subsection{Como tudo Começou}

Por volta de 300 AC, Euclides de Alexandria organizou o conhecimento de sua época acerca da Geometria em uma obra composta por treze volumes, onde construiu, a partir de um pequeno conjunto de axiomas fortemente baseado nos conceitos de pontos e linhas, a chamada \textit{Geometria Euclidiana} \cite{elementosEuclides}. Em contraponto a visão original de Euclides, os primeiros conceitos geométricos usando \textit{apenas distâncias} costumam estar associados aos trabalhos de Heron de Alexandria (10 a 80 DC) \cite{libertiEDG}, com o desenvolvimento de um teorema que leva seu nome, como segue: 
\begin{center}
	\begin{minipage}{0.9 \linewidth}
		\textbf{\textit{Teorema de Heron}:} Sejam $s$ o \textit{semiperímetro} de um triângulo (se $p$ é o perímetro, $s = \frac{p}{2}$) e $a$, $b$ e $c$ os comprimentos dos três lados deste triangulo. Então, a área $A$ do triângulo é
		
		\begin{equation}\tag{Fórmula de Heron}
		A = \sqrt{s(s-a)(s-b)(s-c)}.
		\label{eq:heron}
		\end{equation}
	\end{minipage}
\end{center} 
Pode-se dizer que esse foi o nascimento da Geometria de Distâncias (GD).

Algumas centenas de anos depois, em 1841, Arthur Cayley (1821 a 1895) generalizou a~\ref{eq:heron} através da construção de um determinante que calcula o conteúdo (volume $n$-dimensional) de um \textit{simplex}\footnote{Um simplex é uma generalização do conceito de triangulo a outras dimensões, i.e.: O \textit{0-simples} é um ponto, \textit{1-simplex} é um segmento de reta, \textit{2-simplex} é um triangulo e o \textit{3-simplex} é um tetraedro.} em qualquer dimensão \cite{cayley1841HaronGD}. Um século depois, em 1928, o matemático austríaco Karl Menger (1902 a 1985) re-organizou as ideias de Cayley e trabalhou em uma construção axiomática da geometria através de distâncias \cite{mengerDeterminante} --- donde a alteração no nome do determinante de Cayley para como é conhecido hoje: ``\textit{Determinante de Cayley-Menger}''.  

\begin{center}
	\begin{minipage}{0.9 \linewidth}
		\textbf{Definição:} Sejam $A_0, A_1, \dots, A_n$ $n + 1$ pontos que definem os vértices de um $n$-simplex em um espaço euclidiano $K$-dimensional, onde $n\leq K$, e seja $d_{ij}$ a distância entre os vértices $A_i$ e $A_j$, onde $0\leq i < j \leq n$. Então, o conteúdo $v_n$ desse $n$-simplex é
	\end{minipage}
\end{center} 

\vspace{-0.5cm}

\begin{equation}\tag{Determinante de Cayley-Menger}
v_n^2 = \frac{(-1)^{n+1}}{(n!)^22^n}
\begin{vmatrix}
0 & d^2_{01} &  d^2_{02} & \ldots & d^2_{0n} & 1\\ 
d^2_{01} & 0 & d^2_{12} & \ldots & d^2_{1n} & 1\\ 
d^2_{02} & d^2_{12} &  0 & \ldots & d^2_{2n} & 1\\ 
\vdots & \vdots &  \vdots & \ddots & \vdots & \vdots\\ 
d^2_{0n} & d^2_{1n} & d^2_{2n}  & \ldots & 0 & 1\\ 
1 & 1 & 1  & \ldots & 1 & 0\\ 
\end{vmatrix}.
\label{determinanteCayleyMenger}
\end{equation}

Mas foi só com Leonard Blumenthal (1901 a 1984) que, em 1953, o termo Geometria de Distâncias foi cunhado --- com a publicação de seu livro \textit{``Theory and Applications of Distance Geometry''} \cite{Blumenthal:53}.
Blumenthal dedicou sua vida de trabalho para clarificar, organizar e traduzir as obras originais em alemão \cite{libertiEDG}. Ele acreditava que o problema mais importante nesta área era o \textit{``Problema de Subconjunto''} (ou \textit{Subset Problem}, originalmente), que consistia em encontrar condições necessárias e suficientes a fim de decidir quando uma matriz simétrica era, de fato, uma \textit{matriz de distâncias}\footnote{Seja o par $(\mathcal{X}, d)$ um \textit{espaço métrico} (vide Apêndice~\ref{ap:metric}), onde $\mathcal{X} = \{x_1, \dots, x_n\}$. Uma \textit{matriz de distância sobre $\mathcal{X}$} é uma matriz quadrada $D_{n\times n} = (d_{uv})$ onde, para todo $u,v \leq n$, temos $d_{uv} = d(x_u,x_v)$ \cite{carlileGDandAplications}.} \cite{carlileGDandAplications}. Uma restrição desse problema à métrica euclidiana chama-se \textit{Problema de Matrizes de Distâncias Euclidianas} (ou EDMP, do inglês \textit{Euclidean Distance Matrix Problem}), como segue definida:

\begin{center}
	\begin{minipage}{0.9 \linewidth}
		\label{EDMP}
		\textbf{Problema de Matrizes de Distâncias Euclidianas:} Determinar se, para uma dada matriz quadrada $D_{n\times n} = (d_{ij})$, existe um inteiro $K$ e um conjunto $\{p_1, \dots, p_n \}$ de pontos em $\mathbb{R}^K$ tal que $d_{ij} = \lVert p_i - p_j\rVert$ para todo $i,j \leq n$.
	\end{minipage}
\end{center} 

Condições necessárias e suficientes para que uma matriz seja, de fato, uma matriz de distância euclidiana são dados em \cite{EDMPResolucao}. Para isso, apresenta-se um teorema onde se utiliza o~\ref{determinanteCayleyMenger} na criação de duas condições afirmando que, afim de $D_{n\times n}$ ser uma matriz de distâncias euclidianas, deve haver um $K$-simplex $S$ de referência com conteúdo $v_K \neq 0$ em $\mathbb{R}^K$ e que todos os $(K+1)$-simplex e $(K+2)$-simplex contendo S como uma das faces devem estar contidos em $\mathbb{R}^K$ \cite{carlileGDandAplications}.

Blumenthal percebeu a importância em se respeitar as restrições métricas estabelecidas pelas matrizes de distâncias.
\begin{quotation}
	\textit{Quando temos como dado um conjunto de distâncias entre pares de pontos, a geometria das distâncias pode dar uma dica para encontrar um conjunto de coordenadas correto para pontos no espaço Euclideano tridimensional, satisfazendo as restrições de distâncias dadas.}
	\begin{flushright}
		(Blumenthal, 1953, \cite{Blumenthal:53})
	\end{flushright}
\end{quotation}

Pode-se dizer que resolver o Problema de Matrizes de Distâncias Euclidianas está intimamente relacionado com descobrir as coordenadas dos pontos que definem suas distâncias. Perceba que este é um problema inverso, onde o ``problema direto'' correspondente é calcular distâncias associadas a pares de pontos dados. Note que este estudo tem enorme aplicabilidade \cite{carlileGDandAplications}.
\\

Adiante, em 1979, Yemini (atualmente professor emérito de Ciência da Computação na Universidade de Columbia) foi o primeiro a flexibilizar a definição do EDMP ao considerar um conjunto de distâncias esparso \cite{Yemini:79,carlileGDandAplications} --- i.e., que não se tem todas as distâncias dadas a priori. Com isso, introduziu-se o que se chamou de \textit{Problema Posição - Localização}, onde deseja-se calcular a localização de todos os objetos imersos em um espaço geográfico \cite{Yemini:79}. 

Assim, foi possível re-formular o problema fundamental de Geometria de Distâncias, o qual pode ser caracterizado de forma mais moderna pela utilização da Teoria de Grafos \cite{carlileGDandAplications}.

\subsection{O Problema Fundamental}

Uma \textit{realização} é uma função que mapeia um conjunto de vértices de um grafo $G$ para um espaço euclidiano de alguma dimensão dada \cite{libertiEDG}.

\begin{center}
	\begin{minipage}{0.9 \linewidth}
		\textbf{Problema de Geometria de Distâncias (DGP):} Dados um grafo simples, ponderado e conectado $G = (V, E, d)$ e um inteiro $K>0$, encontre uma realização $x: V \longrightarrow \mathbb{R}^K$ tal que:
		\begin{equation}
		\forall \{u,v\} \in E, \hspace{0.5cm} \lVert x(u) - x(v) \rVert = d(u,v). \label{eq:DGP}
		\end{equation}
	\end{minipage}
\end{center}

Desde que uma realização seja encontrada, também dá-se a ela o nome de \textit{solução} do DGP.

\newpage
\phantomsection
\addcontentsline{toc}{section}{Referências}

\bibliographystyle{unsrt}
\bibliography{references}

\appendix
\newpage
\section{Métricas}
\label{ap:metric}

Como esse texto utiliza fortemente o conceito de distância, é necessário e bem vindo que se gaste algum espaço para uma construção formal dessa ideia. A noção de distância está relacionada com o conceito de \textit{métrica}, como segue.
\\

Seja $\mathcal{X}$ um espaço vetorial $K$-dimensional sobre $\mathbb{R}$. \textit{Métrica} é uma função de dois argumentos que mapeia pares ordenados de elementos em $\mathcal{X}$ para um número real não negativo. Precisamente, para todo $x, y$ e $z$ $\in \mathcal{X}$, uma função $d(\cdot,\cdot): \mathcal{X} \times \mathcal{X} \longrightarrow \mathbf{R}$ é uma métrica se satisfaz os seguintes axiomas:

\begin{enumerate}
	\item $d(x,y) = 0$ se, e somente se, $x = y$; 
	\item $d(x,y) = d(y,x)$;
	\item $d(x,z) \leq d(x,y) + d(y,z)$;
	\item $d(x,y) \geq 0$
\end{enumerate}

Nesse trabalho, quando não é especificado qual métrica se está usando, fica implícita a utilização da \textit{Métrica Euclidiana}, definida em função da \textit{Norma Euclidiana}:

\begin{equation}\tag{Norma Euclidiana}
\forall x, y \in \mathcal{X}, d(x,y) = \lVert x-y \rVert_2 = \sqrt{\langle x, y\rangle} = \sqrt{\sum_{i = 1}^{K} (x_i-y_i)^2}.
\label{eq:normaEuclidiana}
\end{equation}
\\

O par ($\mathcal{X}, d$) é chamado \textit{espaço métrico}. A noção de métrica não depende de espaços vetoriais, donde pode ser facilmente generalizada fazendo $\mathcal{X}$ um conjunto qualquer.
\end{document}

