 \documentclass[a4paper,12pt]{article}
\usepackage[a4paper,top=3cm,bottom=2cm,left=3cm,right=3cm,marginparwidth=1.75cm]{geometry}
\usepackage[brazil]{babel}
\usepackage[T1]{fontenc}
\usepackage[utf8]{inputenc}
\usepackage{amsmath}
\usepackage{MnSymbol}
\usepackage{wasysym}
\usepackage{hyperref}
\usepackage{color}
\definecolor{Blue}{rgb}{0,0,0.9}
\definecolor{Red}{rgb}{0.9,0,0}
\usepackage{esvect}
\usepackage{graphicx}
\usepackage{float}
\usepackage{indentfirst}
\usepackage{caption}
\usepackage{blkarray}
\newcommand\Mark[1]{\textsuperscript#1}
\usepackage{pgfplots}
\usepackage{amsfonts}
\usepackage[english, ruled, linesnumbered]{algorithm2e}
\usepackage{algorithmic}
\newtheorem{definicao}{Definição}[section]
\newtheorem{teorema}{Teorema}[section]

\title{Geometria de Distâncias}
\author{Guilherme Philippi}
\begin{document}
\maketitle
\tableofcontents

\section{Geometria de Distâncias Euclidianas}
Apresenta-se nesta seção uma introdução a \textit{Geometria de Distâncias Euclidianas}. O nome ``Geometria de Distâncias'' diz respeito ao conceito desta ser uma geometria que se baseia em distâncias ao invés de pontos. A palavra ``Euclidiana'' é importante para caracterizar as arestas --- elementos fundamentais associados as distâncias --- como segmentos, sem restringir seus ângulos de incidência \cite{libertiEDG}.

\subsection{Começo}

Os primeiros conceitos geométricos usando apenas distâncias --- em contrapartida com o ponto de vista original de Euclides, por volta de 300 AC, que descreveu a geometria baseado em pontos e linhas \cite{elementosEuclides} --- costumam estar associados aos trabalhos de Heron (10 a 80 DC) \cite{libertiEDG}, com o desenvolvimento de um teorema que leva seu nome, o \textit{Teorema de Heron}: Sejam $s$ o \textit{semiperímetro} de um triângulo (se $p$ é o perímetro, $s = \frac{p}{2}$) e $a$, $b$ e $c$ os comprimentos dos três lados deste triangulo. Então, a área $A$ do triângulo é

\begin{equation}\tag{Fórmula de Heron}
A = \sqrt{s(s-a)(s-b)(s-c)}.
\label{eq:heron}
\end{equation}

Algumas centenas de anos depois, em 1841, Arthur Cayley (1821 a 1895) generalizou a~\ref{eq:heron}, através da construção de um determinante que carrega seu nome \cite{cayley1841HaronGD}.

\phantomsection
\addcontentsline{toc}{section}{Referências}

\bibliographystyle{unsrt}
\bibliography{references}

\end{document}
